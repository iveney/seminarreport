\documentclass[a4paper,12pt]{article}
\usepackage[colorlinks=false,pdfborder=000]{hyperref}
\usepackage[top=1.2in, bottom=1.2in, left=1.2in, right=1.2in]{geometry}
\usepackage{color}
\usepackage{multirow}
\usepackage{titling}
\usepackage{times}
%%%%%%%%%%%%%%%%%%%%%%%%%%%%%%%%%%%%%%%%%%%%%%%%%%%%%%%%%%%%%%%%%%%%%%%%%%%%%%%%%%%%%%%%%%%%%%%%%%%%%%%%%%%%%%%%%%%%%%%%%%%%
\newcommand{\CSE}{\href{http://www.cse.cuhk.edu.hk}{Department of Computer Science and
Engineering}}
\newcommand{\CUHK}{\href{http://www.cuhk.edu.hk}{The Chinese University of Hong Kong}}
\newcommand{\mymail}{\mbox{\textcolor{blue}{\underline{zgxiao@cse.cuhk.edu.hk}}}}
\newcommand{\myname}{\href{http://www.cse.cuhk.edu.hk/~zgxiao}{XIAO Zigang}}
\newcommand{\header}[1]{\noindent {\bf \\#1\\}}

%%%%%%%%%%%%%%%%%%%%%%%%%%%%%%%%%%%%%%%%%%%%%%%%%%%%%%%%%%%%%%%%%%

% modify the font size of title, author and date
\pretitle{\begin{center}\bf \LARGE} \posttitle{\par\end{center}}
\preauthor{\begin{center}
            \small \lineskip 0.5em%
            \begin{tabular}[t]{c}}
\postauthor{\end{tabular}\par\end{center}}
\predate{\begin{center}\small} \postdate{\par\end{center}}

\title{MPhil -- Seminar Report(08 Fall)}
\author{\myname\\\mymail\\\CSE\\\CUHK}
\date{\today}
%%%%%%%%%%%%%%%%%%%%%%%%%%%%%%%%%%%%%%%%%%%%%%%%%%%%%%%%%%%%%%%%%%

\begin{document}
\maketitle

\section{Current issues in Max-SAT solving}
    \begin{tabular}{l l}
    \hline\\
    \bf{Date}:     & September 11, 2008 (Thursday)\\
    \bf{Time}:     & 2:30 p.m. - 3:30 p.m.\\
    \bf{Venue}:    & Room 121, 1/F, Ho Sin-hang Engineering Building,\\
                   & The Chinese University of Hong Kong, Shatin, N.T.\\
    \bf{Speaker}:  & Prof. Javier Larrosa, Associate Professor, \\
                   & Universitat Politecnica de Catalunya, Barcelona, Spain \\
    \bf{Organizer}:& Department of Computer Science and Engineering, CUHK \\\\
    \hline\\
    \end{tabular}

\header{Abstract}

SAT is the most famous NP-complete problem. Given a boolean
formula in conjunctive normal form, the goal is to find an
assignment that satisfies all its clauses. Max-SAT is the
optimization version of SAT and the goal is to maximize the number
of satisfied clauses. Many optimization problems arising in
scheduling, bioinformatics, electronic markets, circuit design and
many other domains can be naturally modeled as Max-SAT.

In the last 5 years Max-SAT solvers have evolved from the
implementation of very naive ideas to modern and sophisticated
pieces of software. The main reason of the breakthrough is the
development of a true Max-SAT logic. In this talk I will present
such a logic and review how it is exploited by modern Max-SAT
solvers.

\header{Comment}

As we all know that SAT and Max-SAT is NP-complete problem which
computer scientists and mathematicians work for years to find better
solutions. In this talk, professor Larrosa first introduced SAT
problem and Max-SAT, and some other useful concepts, such as
Unsatisfiable Core, Unit Propagation,etc. . The speaker introduced
three approaches, which are efficient to some different extent. They
are:
\begin{itemize}
    \item{Translate Max-SAT to SAT and solve}
    \item{Use inference resolution}
    \item{Use search technique with lower bound and incrementally}
\end{itemize}

All these methods inspired me a lot during the talk. Although I am
not so familiar with the topic at first, I grasped some essence of
the problem as well as modern solve techniques from this talk.
However, the time is a little bit limited that the speaker failed to
introduce the future research trend, which I am also very interested
in.

%%%%%%%%%%%%%%%%%%%%%%%%%%%%%%%%%%%%%%%%%%%%%%%%%%%%%%%%%%%%%%%%%%%%%

\section{Iterative Relaxation}

    \begin{tabular}{l l}
    \hline\\
    \bf{Date}:     & September 11, 2008 (Thursday)\\
    \bf{Time}:     & 4:30 p.m. - 5:30 p.m.\\
    \bf{Venue}:    & Room 833, 1/F, Ho Sin-hang Engineering Building,\\
                   & The Chinese University of Hong Kong, Shatin, N.T.\\
    \bf{Speaker}:  & Prof. Lap Chi Lau, Assistant Professor, \\
                   & Department of Computer Science and Engineering, CUHK \\
    \bf{Organizer}:& Department of Computer Science and Engineering, CUHK \\\\
    \hline\\
    \end{tabular}

\header{Abstract}

In this talk we will demonstrate an iterative relaxation method as a
framework to analyze linear programming formulations of
combinatorial optimization problems. This method was first developed
to tackle degree-bounded network design problems, where it gave
approximation algorithms with only additive constant errors. In
particular, it gave an approximation algorithm with error at most 1
for the minimum bounded degree spanning tree problem, proving a
conjecture of Goemans. Then we will present the recent developments
of this method. We show how this method provides new proofs of exact
linear programming formulations for many classical problems, and see
how these new proofs lead us to obtain new results in approximation
algorithms. Finally we will discuss potential future directions of
this method, and highlight some interesting open problems.  This
iterative relaxation method is an extension of Jain's iterative
rounding method.  We will start this talk with Jain's method and the
necessary background.

Joint work with T. Kiraly, J. Naor, M. Salavatipour, R. Ravi, and M.Singh.

\header{Comment}

Professor Lau introduced iterative relaxation method, which used to
study LP for combinatorial optimization problems. This method is
very efficient and it seems that it is a good research topic. The
last part of the talk is about open problems, which are very
interesting. Prof. Lau has impressed everyone present with this
method.

%%%%%%%%%%%%%%%%%%%%%%%%%%%%%%%%%%%%%%%%%%%%%%%%%%%%%%%%%%%%%%%%%%%%%

\section{Workshop on Computer Aided Design }
    \begin{tabular}{l l}
    \hline\\
    \bf{Date}:     & October 8, 2008 (Wednesday)\\
    \bf{Time}:     & 9:00 a.m. - 12:30 p.m.\\
    \bf{Venue}:    & Council Chamber,the University of Hong Kong\\
    \bf{Speaker}:  & Prof. C. K. Cheng, U.C. San Diego\\
                   & Prof. Massoud Pedram, USC \\
                   & Prof. Tim Cheng, U.C. Santa Barbara \& U. of Tokyo\\
    \bf{Organizer}:& Department of Electrical and Electronic Engineering, HKU \\\\
    \hline\\
    \end{tabular}

\header{Abstract}
\begin{enumerate}
  \item \textbf{Breaking the Wall of Interconnect for Digital System Performance}\\
  As technology scales, interconnects become one of the most critical factors
  in determining the digital system speed and power consumption.
  Transmission lines offer the potential to break the wall that blocks the interconnect performance.
  First, the transmission line can allow the signal to travel at the speed of light in the medium.
  Second, the signal toggles as wave instead of enforced electronic charges and thus saves power.

  \item \textbf{Re-engineering CMOS VLSI Circuits for Low Power}\\
  Today��s rising power densities in VLSI circuits constitute the primary challenge to continued CMOS scaling.
  In my talk, I will first describe state-of-the-art in power-aware design methodologies and tools.
  Next, I will provide a list of the critical new VLSI design and electronic design automation capabilities
  which are needed in order to enable high performance design under a power density constraint.
  Finally I will present a practical lower bound on the energy dissipation per operation in CMOS VLSI circuits
  and discuss what it tells us about the current state of the technology and design.

  \item \textbf{Design for Reliability and Robustness}\\
  Future hardware systems must have sufficient robustness to cope with failures resulting
  from the increasing variability and reliability concerns. This requirement not only applies
  to high-end systems but also becomes a necessity for consumer electronics.
  Failures due to design bugs, manufacturing defects and variations,
  and environmental noise are becoming facts to be dealt with, not just problems to be solved.
\end{enumerate}

\header{Comment}

Prof. C. K. Cheng present a transmission line approach to improve communication speed with significant power reduction.
An optimization flow is developed based on eye-diagram prediction and sequential quadratic programming.
The scalability of the scheme is investigated for technology nodes from 90nm to 22nm.

Prof. Massoud Pedram gave us a well-organized talk about the state of art of power-saving techniques and technologies.
I learned much from it.

Prof. Tim Cheng reminds us that verification, test,
and fault tolerance technologies all play critical roles in designing robust and reliably systems.
Built-in redundancy to tolerate errors or built-in self-recovery from errors will become necessary
to ensure sufficient system yield and reliability.
Designing a robust system with spares and self-reconfiguration capability could also alleviate
the need for burn-in/stress test in the manufacturing line, which has become extremely costly.

%%%%%%%%%%%%%%%%%%%%%%%%%%%%%%%%%%%%%%%%%%%%%%%%%%%%%%%%%%%%%%%%%%%%%

\section{Innovation: An MIT CSAIL Perspective}
    \begin{tabular}{l l}
    \hline\\
    \bf{Date}:     & October 31, 2008 (Friday)\\
    \bf{Time}:     & 4:45 p.m. - 6:15 p.m.\\
    \bf{Venue}:    & T.Y. Wong Hall, 5/F, Ho Sin Hang Engineering Building,\\
                   & The Chinese University of Hong Kong, Shatin, N.T.\\
    \bf{Speaker}:  & Professor Victor ZUE,\\
                   & Massachusetts Institute of Technology, USA \\
    \bf{Organizer}:& Department of Computer Science and Engineering, CUHK \\\\
    \hline\\
    \end{tabular}

\header{Abstract}

For more than four decades, the MIT Computer Science and Artificial Intelligence Laboratory (CSAIL)
and its predecessors the Artificial Intelligence Laboratory and the Laboratory for Computer Science,
have contributed many technical innovations, ranging from time-sharing and RSA public key encryption
to robotics and human-like interfaces. Some of these innovations have spawned successful start-ups
or have been acquired by multinationals. In this talk, I would like to offer my personal opinion about
the factors contributing to this innovation-rich research environment. I will illustrate my points with
examples drawn from past and current research.

\header{Comment}

Professor ZUE made a perfect talk to us. He introduced to us the information of CSAIL,
presented many interesting application in CSAIL, and shared the experience about innovation with us.
What impressed me most is his humorous and well-prepared slides. This really attracts lots of audience.
However, due to time limit, I missed the chance to ask my questions.

%%%%%%%%%%%%%%%%%%%%%%%%%%%%%%%%%%%%%%%%%%%%%%%%%%%%%%%%%%%%%%%%%%%%%

\section{The Internet is Flat: a Brief History of Networking in the Next Ten Years}
    \begin{tabular}{l l}
    \hline\\
    \bf{Date}:     & November 14, 2008 (Friday) \\
    \bf{Time}:     & 10:30 a.m.- 11:30 a.m. \\
    \bf{Venue}:    & Rm.121, Ho Sin Hang Engineering Building,\\
                   & The Chinese University of Hong Kong, Shatin, N.T.\\
    \bf{Speaker}:  & Prof. Don Towsley, Department of Computer Science,\\
                   & University of Massachusetts\\
    \bf{Organizer}:& Department of Computer Science and Engineering, CUHK \\\\
    \hline\\
    \end{tabular}

\header{Abstract}

The current Internet consists of ten to twenty thousand different interconnected autonomous networks.
In many cases these networks have negotiated cumbersome bilateral and multilateral agreements that
constrain how data is allowed to flow from source to destination.
For example, universities can communicate with each other through the Abilene network but
must rely on other networks to communicate with non-academic entities such as Google.
These agreements generally impose a loose hierarchy on the Internet with
respect to the flow of data and information.
The recent development of peer-to-peer file sharing technology,
however, has the unintended effect of relaxing and voiding these agreements.
This has resulted in a "flattening" of the Internet.
In this talk we review the introduction of peer-to-peer (p2p) technology and
examine the implications that it may have on the Internet over the next ten years.
In particular, we examine the effects of p2p on economics for Internet service providers (ISPs),
and the impact on how they manage and engineer their networks. We focus on one p2p technology,
``swarming," as exemplified by BitTorrent,
and examine how it could further flatten the Internet if it were to become the basis of a ``universal swarm" and
form the basis of a new data transfer architecture over the next ten years.
Last, we present a research agenda centered on swarm technology to make this happen.
We will focus in particular on interesting theoretical and algorithmic challenges that will arise with such an architecture.

\header{Comment}

The model Prof. Don Towsley proposed in this talk is very promising.
I also agree that the Internet will become more and more decentralized.
However, I think there are still some points we should not neglect:
\begin{enumerate}
  \item The bandwidth of the network and the ISP's benefit-driven nature.
        We have to admit that today's network still cannot bear such a great data flow
        induced by p2p technology, and ISP's will not be happy in such an Internet model,
        since they will have to pay much but can not gain more.
  \item The cheating, security, privacy and legal issues.
        There are already such issues in p2p network.
        It becomes very crucial to handle these problems.
\end{enumerate}

Nevertheless, we have to admit that the world will become flatter and flatter with the technology advances.

%%%%%%%%%%%%%%%%%%%%%%%%%%%%%%%%%%%%%%%%%%%%%%%%%%%%%%%%%%%%%%%%%%%%%

\section{The public project problem}
    \begin{tabular}{l p{.8\linewidth}}
    \hline\\
    \bf{Date}:     & November 14, 2008 (Friday) \\
    \bf{Time}:     & 2:00 p.m. - 3:00 p.m. \\
    \bf{Venue}:    & Rm.121, 1/F, Ho Sin Hang Engineering Building,\\
                   & The Chinese University of Hong Kong, Shatin, N.T.\\
    \bf{Speaker}:  & Professor Krzysztof R. Apt, the Institute for Logic, Language
                     and Computation University of Amsterdam The Netherlands\\
    \bf{Organizer}:& Department of Computer Science and Engineering, CUHK \\\\
    \hline\\
    \end{tabular}

\header{Abstract}

In the public project problem a decision must be made whether to
construct a discrete public good (say a bridge) without knowledge
of the utility functions of the interested players. This is a
well-known example of a mechanism design problem. The classic
pivotal mechanism provides a solution but, unavoidably, at the
cost of taxes players have to pay.  We explain that this solution
is optimal, ..., well, sometimes.  Additionally, when the players
are allowed to move sequentially, a better solution turns out to
be optimal.

The lecture will start with a short introduction to the mechanism
design, the subject of the most recent Nobel prize in Economics.
(Based on recent works with V. Conitzer, A. Estevez-Fernandez, V.
Markakis and M. Guo).

\header{Comment}

Public project problem is an interesting problem,
which is being studied by Prof. K.R.Apt.
He first introduced this problem, and previous study of it.
And then he presented to us his newest research result that his theorem
is optimal under sequential move. This sounds good.
However it is under perfect assumption. Circumstance may be:
\begin{itemize}
  \item Not every one knows this strategy maximize its utility(\textbf{optimal}).
  \item There may be cheating in the game, e.g. \textbf{collusion}.
\end{itemize}

For the second point, no one can answer it now, and Prof. Apt is currently studying in it.

%%%%%%%%%%%%%%%%%%%%%%%%%%%%%%%%%%%%%%%%%%%%%%%%%%%%%%%%%%%%%%%%%%%%%

\end{document}
